% !TEX root = ./POLS_1600_syllabus_Spring_2025.tex

\part*{Overview}

\section{General Information}
\subsection{Course Website} \url{https://pols1600.paultesta.org}
\subsection{Canvas} \url{https://canvas.brown.edu/courses/1098409}
\subsection{Zoom} \url{https://brown.zoom.us/j/97089459564}
\subsection{Where/When}
We meet Tuesdays and Thursdays from 10:30--11:50 am in \href{https://brown.edu/Facilities/Facilities_Management/maps/index.php#building/SMITH-BUON}{Smith-Buonanno Hall G01}.

\subsection{Office Hours}
 Tuesdays from \textbf{1:00 pm--3:00 pm} at 111 Thayer St Room 339. If possible, please reserve a spot \href{https://calendar.app.google/31gV2U8fHNmv9LsV6}{here}. I am also generally available to meet 

\section{Course Description}

This class is an introduction to applied statistics as practiced in political science. It is computing intensive, and, as such, will enable students to execute basic quantitative analyses of social science data using the linear model with statistical inference arising from re-sampling and permutation based techniques as applied through statistical computing language \href{https://cran.r-project.org/}{R} with \href{https://www.rstudio.com/}{RStudio}. By the end of the course, a successful student will be able to find social science data online, download it, analyze it, and write about how the analyses bear on focused social science or policy questions.


\section{Expectations}

More than anything I assume a willingness to engage with mathematics, data analysis, computer programming, and the practice of social science thinking and writing. I also assume you've taken at least one class in algebra at the level taught in most high schools in the United States and have used a personal computer to read and type email and other documents and have some experience with the Internet. 

I also assume that you will read the syllabus and that you keep up to date on changes in the syllabus which will be announced in class. You should not expect a response to emails that ask a question already answered in the syllabus. 

This is an experimental class so you should expect that the syllabus will change throughout the term. Make sure you have the syllabus with the latest date stamp. I will announce syllabus changes via the emails sent from Canvas. 


\section{Academic Integrity}

Neither the University nor I tolerate cheating or plagiarism. The Brown Writing Center defines plagiarism as ``appropriating another person's ideas or words (spoken or written) without attributing those word or ideas to their true source.'' The consequences for plagiarism are often severe, and can include suspension or expulsion. This course will follow the guidelines in the Academic Code for determining what is and isn't plagiarism:

\begin{quote}
In preparing assignments a student often needs or is required to employ outside sources of information or opinion. All such sources should be listed in the bibliography. Citations and footnote references are required for all specific facts that are not common knowledge and about which there is not general agreement. New discoveries or debatable opinions must be credited to the source, with specific references to edition and page even when the student restates the matter in his or her own words. Word-for-word inclusion of any part of someone else's written or oral sentence, even if only a phrase or sentence, requires citation in quotation marks and use of the appropriate conventions for attribution. Citations should normally include author, title, edition, and page. (Quotations longer than one sentence are generally indented from the text of the essay, without quotation marks, and identified by author, title, edition, and page.) Paraphrasing or summarizing the contents of another's work is not dishonest if the source or sources are clearly identified (author, title, edition, and page), but such paraphrasing does not constitute independent work and may be rejected by the instructor. Students who have questions about accurate and proper citation methods are expected to consult reference guides as well as course instructors.
\end{quote}


We will discuss specific information about your written work in class in more detail, but if you are unsure of how to properly cite material, please ask for clarification. If you are having difficulty with writing or would like more information or assistance, consult the Writing Center, the Brown library and/or the \href{https://www.brown.edu/academics/college/degree/policies/academic-code}{Academic Code} for more information.

You may use generative AI tools such as ChatGPT to help troubleshoot your code. Such tools can only take you so far, and without learning to code yourself, you will hit walls and limits and that ChatGPT cannot solve for you. If it becomes apparent that you are not actually engaging labs and assignments but simply asking ChatGPT to solve these problems for you, you will receive no credit for that assignment and we will schedule a meeting during office hours to discuss your performance in class.


\section{Community Standards}

All students and the instructor must be respectful of others in the classroom. If you ever feel that the classroom environment is discouraging your participation or problematic in any way, please contact me.

\section{Accessibility}
Brown University is committed to full inclusion of all students. Please inform me if you have a disability or other condition that might require accommodations or modification of any of these course procedures. You may speak with me after class or during office hours. For more information contact Student and Employee Accessibility Services at 401-863-9588 or SEAS@brown.edu.

\subsection{Academic Accommodations}
Any student with a documented disability is welcome to contact me as early in the semester as possible so that we may arrange reasonable accommodations. As part of this process, please be in touch with Student Accessibility Services by calling 401-863-9588 or \href{http://brown.edu/Student_Services/Office_of_Student_Life/seas/index.html}{online}

\subsection{Diversity and Inclusion}

This course is designed to support an inclusive learning environment where diverse perspectives are recognized, respected and seen as a source of strength. It is my intent to provide materials and activities that are respectful of various levels of diversity: mathematical background, previous computing skills, gender, sexuality, disability, age, socioeconomic status, ethnicity, race, and culture. Toward that goal:

\begin{itemize}
\item If you have a name and/or set of pronouns that differ from those that appear in your official Brown records, please let me know!
\item If there are things going on inside or outside of class that are affecting your performance in class, please don't hesitate to talk to me, provide anonymous feedback through our course survey, or \href{https://www.brown.edu/academics/college/speak-academic-dean}{contact} one of Brown's Academic Deans.
\end{itemize}
