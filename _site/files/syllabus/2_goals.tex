% !TEX root = ./POLS_1600_syllabus_Spring_2024.tex

\section{Course Goals}
In the bestiary of the social sciences, methodological training typically follows either the path of the tortoise or the hare. There`s no right way to run this race. Going slow and steady ideally provides you with a foundation to learn the methods you need to know. The danger of this approach is that you can spend so much time up front doing proofs and problem sets that you lose sight of why you wanted to obtain this training in the first place. Similarly, ranging far and wide can provide an overview of the toolkit available you, but without a strong foundation in the motivations and assumptions behind these methods, there's a risk that you'll end up using a expensive table saw when a simple wrench would have sufficed.

This course aims to strike a middle ground. To continue the (belabored) animal metaphor, we'll start off as hedgehogs, focusing on knowing a few things well: inference (descriptive, statistical, and causal), linear models (as a tool for inference), and extensions and alternatives to the linear model (to facilitate better inferences). By the end of the course, we'll be ready to blossom (mutate?) into methodological foxes capable of learning the many things skills and methods needed for our research.

\section{Requirements}
To accomplish this metamorphosis, we'll need the following:

\begin{itemize}
\item Some math
\item Some programming and computing skills
\item Some general life skills
\end{itemize}

\subsection{Math}

You either already know, or will learn, all the math you need to take this course.\footnote{This is not the same as all the math you need to know be a successful, methodologically sophisticated political scientist. But it's a start, and one that will hopefully help you figure out what additional training you'll need.} We'll go over some key theorems of probability and statistics in class, emphasizing conceptual understanding (often illustrated via simulation) over formal proofs.\footnote{We'll do the proofs as well, but your focus should be on making sure you understand concepts and implications rather than specific derivations}. Along the way, we'll need some calculus and linear algebra to make our lives easier, and so we'll briefly review this material together in class.

\subsection{Computing}

Doing quantitative, empirical social science research requires working with data. Today, working with data requires a computer and statistical software. I assume that you have, or will acquire, a laptop that you will bring to class. In terms of software, there are many possible options. In this class,\texttt{R}.\footnote{Available for free at \url{https://cran.r-project.org/}. Python is also increasingly common among social scientists}. 

All the slides, notes, and assignments in this class are produced using R Markdown, a free, open-source tool for creating reproducible research. \textbf{All of your assignments and papers for this class will also be submitted using R Markdown.}  It's a short but steep learning curve, the benefits of which (pretty documents, nicely formatted tables and figures, easy integration with citation managers) far outweigh the costs (finicky syntax)

\subsection{General}  

Like any course, success in this class requires preparation, participation and perseverance. In terms of preparation, I expect that you will have done the readings and submitted your assignments on time (more on that below). In short, you'll get out of this class what you put in. In terms of participation, I expect that you will come to class eager to learn and engage with that week's topics. If you have a question, ask it. If you're getting an error, share it. In some ways, your job is to make errors. To paraphrase Joyce: people of genius make no mistakes. Our errors are volitional and portals of discovery. While this experience can be challenging and frustrating, it is also incredibly rewarding. I fully expect you persevere through the problems and difficulties that inevitably arise in this course, and will do everything I can to help in this process.


