% !TEX root = ./POLS_1600_syllabus_Spring_2023.tex
\part*{Course Structure and Policies}

\section{Class}

% Broadly the structure for each class is as follows. Before each class on Tuesday, you will have done the assigned readings for the week and submitted the prior week's lab (more on that below) to Canvas \textbf{by 9:00 pm on the Sunday before class}. On Tuesday, I'll post solutions to the prior assignment.  During class on Tuesday, we'll review the prior weeks' work, discuss the current week's topic and then get started together on that week's lab. Whatever you don't finish in class you will be expected to complete and submit to the Canvas by 9:00 pm on the Sunday after class. Course slides, assignments, comments, and supplemental material will be posted to Canvas.

This course meets two times a week for 80 minutes on Tuesdays and Thursdays. Tuesday's class will be devoted to lecture, demonstration and review. Recorded versions of these lectures will be provided on Canvas after class. Thursday's class will focus on applications of these concepts through brief labs where you'll work with real data from a variety of sources. I assume that you will come to class having done each week's assigned readings and reviewed material from the previous week's lectures and labs. Slides and labs are available on Canvas and \url{https://pols1600.paultesta.org}

\section{Attendance}

You may miss two classes without it having any effect on the attendance portion of your grade. After two absences, each additional absence (without a written note from the University) will reduce your final grade by 1 percent.


\section{Readings}

There is one required textbooks for the course (\textbf{Estimated cost: $\sim$\$38.50} for the ebook, \$55.00 for the paperback):

\textbf{\bibentry{Imai2022-pm}} 

The primary textbook on which the course is structured. Most chapters are spread over multiple weeks. You should read this text with your laptop and R Studio open. Execute the code in the main text and ideally try to complete the assignments and exercises at the end of the chapter.\footnote{Seriously. Working carefully through these examples will be incredibly helpful and rewarding. If you're taking the time to read this footnote, send me picture of a cute animal and I'll add 1 point of extra credit to your final paper grade. See, your hard work is already paying off.} 

Additional readings will be listed below and available to download on the course website and Canvas. 


\section{Labs}

The bulk of the work and learning you'll do in the course comes in the form of \textbf{weekly labs} in which you'll explore a given data set or paper using R. You'll be given an R Markdown document that will guide you through a set of exercises to teach concepts covered in the lectures and reading. You'll code in \texttt{R} and summaries of your findings in R Markdown. You will compile your document to produce an html document which you will \textbf{submit on Canvas by the end of each class.}

All work in this class \textbf{MUST BE SUBMITTED ONLINE VIA CANVAS.}

You will work in groups on these labs. One member of your group will submit a lab. One question from the lab will be randomly selected for grading. 

\section{Tutorials}

In addition to weekly labs, you will also work through \textbf{weekly tutorials} made available to you through the `qsslearnr' package. These tutorials provide you with an opportunity to practice your programming and review concepts from the text and lecture. After completing each tutorial, you will download your progress report and upload this file to Canvas by midnight on Friday each week a tutorial is assigned. If you upload a report by Friday, you receive a grade of 100\% on that Tutorial. If you upload a report after Friday, you receive a grade of 50\%. If you do not upload a report, you receive a grade of 0\%. There are 11 total tutorials for the course. Your lowest grade on the Tutorials will be dropped.

These Tutorials are for your personal benefit. You may collaborate with peers, but you must submit your own file.

\section{Assignments} 

In addition to weekly labs, you will complete periodic group assignments developing an original research presentation applying skills you have learned in this class to a topic of your choosing. All assignments are due the Friday after the class with which they are associated. 

The timeline of assignments for your final paper is as follows:

\begin{itemize}
\item \textbf{Week 3: Research Topics}
\item \textbf{Week 6: Identifying Datasets}
\item \textbf{Week 8: Data Explorations}
\item \textbf{Week 11: Draft of Research Presentation}
\item \textbf{Week 12: Research Presentations}
\item \textbf{Week 13: Final Paper}
\end{itemize}

Assignments must be submitted on time to Canvas. No late work will be accepted without prior approval of the instructor or a note from the University.


\section{Grades}

Your final grade for this course will be calculated as follows:

\begin{itemize}
	\item \textbf{5\% Class attendance}
	\item \textbf{10\% Class involvement and participation}
	\item \textbf{10\% Tutorials}
	\item \textbf{30\% Labs}
	\item \textbf{20\% Assignments not including the final Paper}
	\item \textbf{25\% Final Project}
\end{itemize}


Labs, assignments excluding the final presentation, will be graded graded out of 100 roughly on a \checkmark + (100, completed on time, acceptable), \checkmark (85, completed on time, passable), \checkmark - (0 not submitted on time, unacceptable). The lowest three lab grades will be dropped from your final lab grade. Tutorials are graded on pass (submitted on time = 100\% ) - fail (not submitted =0) based submitting your progress report from the tutorial by Friday each week. If you submit a Tutorial after the week it's do, you will receive partial credit (50\%). Your final projects will be graded on 100-point scales with rubrics provided beforehand.


\textbf{Incomplete Work} Assignments not turned in will be counted as zero in the calculation of the final grade.

\textbf{Computers in class} Please bring your laptops if you have them. We will install \href{https://cran.r-project.org/}{R} and \href{https://www.rstudio.com/products/RStudio/#Desktop}{RStudio} together. If you do not own a laptop, you can still work in a group of other people who have laptops and will be able to complete the in-class worksheets without a problem. In fact, it is ideal if each group of 2-4 people works with one laptop and then shares the work among themselves. Of course, feel free to work on your own outside of class. 

\section{Time}

This course meets 27 times over 13 weeks in the semester. Each class is 80 minutes long so you should expect to spend approximately 36 hours total in class; approximately 4 hours per week reading the textbook and reviewing material (42 hours total);  approximately 22 hours on tutorials each week, approximately 30 hours on assignments for the final paper; approximately 50 hours researching, writing, and revising your final presentation; and at least .5 hours meeting with me in person to discuss your work (Estimated Total Time: 180.5 hours)




